\documentclass[graybox, envcountchap]{styles/svmult}
\usepackage{CJKutf8}
\author{lijiankoucoco@163.com}
\title{Machine Learning}
\usepackage{amssymb,amsmath,bm}
\DeclareMathAlphabet{\mathcal}{OMS}{cmsy}{m}{n}
\usepackage{textcomp}
\newcommand\abs[1]{\left\lvert#1\right\rvert}
\usepackage{longtable}
\usepackage{ulem}
\usepackage{algorithm2e}
\usepackage{tocbibind}
\renewcommand{\bibname}{References}
%\usepackage{mathptmx}        % selects Times Roman as basic font
\usepackage{helvet}          % selects Helvetica as sans-serif font
\usepackage{courier}         % selects Courier as typewriter font
\usepackage{makeidx}         % allows index generation
\usepackage{graphicx}        % standard LaTeX graphics tool
\usepackage{multicol}        % used for the two-column index
\usepackage[bottom]{footmisc}% places footnotes at page bottom
\usepackage[bookmarksnumbered=true,
            bookmarksopen=true,
            colorlinks=true,
            linkcolor=blue,
            anchorcolor=blue,
            citecolor=blue
           ]{hyperref}

\makeindex             % used for the subject index

\begin{document}
\begin{CJK}{UTF8}{gkai}
\frontmatter%%%%%%%%%%%%%%%%%%%%%%%%%%%%%%%%%%%%%%%%%%%%%%%%%%%%%%
\include{titlepage}
\include{preface}
\tableofcontents
\Extrachap{Notation}

\section*{Introduction}
It is very difficult to come up with a single, consistent notation to cover the wide variety of
data, models and algorithms that we discuss. Furthermore, conventions difer between machine
learning and statistics, and between different books and papers. Nevertheless, we have tried
to be as consistent as possible. Below we summarize most of the notation used in this book,
although individual sections may introduce new notation. Note also that the same symbol may
have diferent meanings depending on the context, although we try to avoid this where possible.


\section*{General math notation}

\begin{longtable}{p{2.4cm}p{8.9cm}}
Symbol & Meaning \\
\noalign{\smallskip}\hline\noalign{\smallskip}
$\lfloor x \rfloor$ & Floor of $x$, i.e., round down to nearest integer\\
$\lceil x \rceil$ & Ceiling of $x$, i.e., round down to nearest integer\\
$\vec{x} \otimes \vec{y}$ & Convolution of $\vec{x}$ and $\vec{y}$\\
$\vec{x} \odot \vec{y}$ & Hadamard (elementwise) product of $\vec{x}$ and $\vec{y}$\\
$a \wedge b$ & logical AND\\
$a \vee b$ & logical OR\\
$\neg a$ & logical NOT\\
$\mathbb{I}(x)$ & Indicator function, $\mathbb{I}(x)=1$ if x is true, else $\mathbb{I}(x)=0$\\
$\infty$ & Infinity\\
$\rightarrow$ & Tends towards, e.g., $n \rightarrow \infty$\\
$\propto$ &Proportional to, so $y = ax$ can be written as $y \propto x$\\
$\abs{x}$ & Absolute value\\
$\abs{\mathcal{S}}$ & Size (cardinality) of a set\\
$n!$ & Factorial function\\
$\nabla$ & Vector of first derivatives\\
$\nabla^2$ & Hessian matrix of second derivatives\\
$\triangleq$ & Defined as\\
$O(\cdot)$ & Big-O: roughly means order of magnitude\\
$\mathbb{R}$ & The real numbers\\
$1:n$ & Range (Matlab convention): $1:n = {1, 2,...,n}$\\
$\approx$ & Approximately equal to\\
$\arg\max\limits_x f(x)$ & Argmax: the value $x$ that maximizes $f$\\
$B(a,b)$ & Beta function, $B(a,b)=\dfrac{\Gamma(a)\Gamma(b)}{\Gamma(a+b)}$\\
$B(\vec{\alpha})$ & Multivariate beta function, $\dfrac{\prod\limits_k \Gamma(\alpha_k)}{\Gamma(\sum\limits_k \alpha_k)}$\\
$\binom{n}{k}$ & $n$ choose $k$ , equal to $n!/(k!(n−k )!)$\\
$\delta(x)$ & Dirac delta function,$\delta(x)=\infty$ if $x=0$, else $\delta(x)=0$\\
$exp(x)$ & Exponential function $e^x$\\
$\Gamma(x)$ & Gamma function, $\Gamma(x)=\int_0^\infty u^{x-1}e^{-u}du$\\
$\Psi(x)$ &  Digamma function,$Psi(x)=\dfrac{d}{dx}\log\Gamma(x)$\\
$\mathcal{X}$ & A set from which values are drawn (e.g.,$\mathcal{X}=\mathbb{R}^D$)\\
\noalign{\smallskip}\hline\noalign{\smallskip}
\end{longtable}


\section*{Linear algebra notation}
We use boldface lower-case to denote vectors, such as $\vec{x}$, and boldface upper-case to denote matrices, such as $\vec{X}$. We denote entries in a matrix by non-bold upper case letters, such as $X_{ij}$. Vectors are assumed to be column vectors, unless noted otherwise.

\begin{longtable}{p{2.4cm}p{8.9cm}}
Symbol & Meaning \\
\noalign{\smallskip}\hline\noalign{\smallskip}
$\vec{X} \succ 0$ & $\vec{X}$ is a positive definite matrix\\
$tr(\vec{X})$ & Trace of a matrix\\
$det(\vec{X})$ & Determinant of matrix $\vec{X}$\\
$\abs{\vec{X}}$ & Determinant of matrix $\vec{X}$\\
$\vec{X}^{-1}$ & Inverse of a matrix\\
$\vec{X}^{\dagger}$ & Pseudo-inverse of a matrix\\
$\vec{X}^T$ & Transpose of a matrix\\
$\vec{x}^T$ & Transpose of a vector\\
$diag(x)$ & Diagonal matrix made from vector $\vec{x}$\\
$diag(X)$ & Diagonal vector extracted from matrix $\vec{X}$\\
$\vec{I}$ or $\vec{I}_d$ & Identity matrix of size $d \times d$ (ones on diagonal, zeros of)\\
$\vec{1}$ or $\vec{1}_d$ & Vector of ones (of length $d$)\\
$\vec{0}$ or $\vec{0}_d$ & Vector of zeros (of length $d$)\\
$\abs{\abs{\vec{x}}}=\abs{\abs{\vec{x}}}_2$ & Euclidean or $\ell_2$ norm $\sqrt{\sum\limits_{j=1}^{d} x_j^2}$\\
$\abs{\abs{\vec{x}}}_1$ & $\ell_1$ norm $\sum\limits_{j=1}^{d} \abs{x_j}$\\
$\vec{X}_{:,j}$ & j’th column of matrix\\
$\vec{X}_{i,:}$ & transpose of i’th row of matrix (a column vector)\\
$\vec{X}_{i,j}$ & Element $(i,j)$ of matrix $\vec{X}$ \\
$\vec{x} \otimes \vec{y}$ & Tensor product of $\vec{x}$ and $\vec{y}$\\
\noalign{\smallskip}\hline\noalign{\smallskip}
\end{longtable}


\section*{Probability notation}
We denote random and fixed scalars by lower case, random and fixed vectors by bold lower case, and random and fixed matrices by bold upper case. Occasionally we use non-bold upper case to denote scalar random variables. Also, we use $p()$ for both discrete and continuous random variables

\begin{longtable}{p{2.4cm}p{8.9cm}}
Symbol & Meaning \\
\noalign{\smallskip}\hline\noalign{\smallskip}
$X,Y$ & Random variable\\
$P()$ & Probability of a random event\\
$F()$ & Cumulative distribution function(CDF), also called distribution function\\
$p(x)$ & Probability mass function(PMF)\\
$f(x)$ & probability density function(PDF) \\
$F(x,y)$ & Joint CDF\\
$p(x,y)$ & Joint PMF \\
$f(x,y)$ & Joint PDF\\
$p(X|Y)$ & Conditional PMF, also called conditional probability\\
$f_{X|Y}(x|y)$ & Conditional PDF\\
$X \perp Y$ & X is independent of Y\\
$X \not\perp Y$ & X is not independent of Y\\
$X \perp Y | Z $ & X is conditionally independent of Y given Z\\
$X \not\perp Y | Z $ & X is not conditionally independent of Y given Z\\
$X \sim p$ & X is distributed according to distribution $p$\\
$\vec{\alpha}$ & Parameters of a Beta or Dirichlet distribution\\
$cov[X]$ & Covariance of X\\
$\mathbb{E}[X]$ & Expected value of X\\
$\mathbb{E}_q[X]$ & Expected value of X wrt distribution $q$\\
$\mathbb{H}(X)$ or $\mathbb{H}(p)$ & Entropy of distribution $p(X)$\\
$\mathbb{I}(X;Y)$ & Mutual information between X and Y\\
$\mathbb{KL}(p||q)$ & KL divergence from distribution $p$ to $q$\\
$\ell(\vec{\theta})$ & Log-likelihood function\\
$L(\theta,a)$ & Loss function for taking action $a$ when true state of nature is $\theta$\\
$\lambda$ & Precision (inverse variance) $\lambda=1/\sigma^2$\\
$\Lambda$ & Precision matrix $\Lambda=\Sigma^{-1}$\\
$mode[\vec X]$ & Most probable value of $\vec X$\\
$\mu$ & Mean of a scalar distribution\\
$\vec{\mu}$ & Mean of a multivariate distribution\\
$\Phi$ & cdf of standard normal\\
$\phi$ & pdf of standard normal\\
$\vec{\pi}$ & multinomial parameter vector, Stationary distribution of Markov chain\\
$\rho$ & Correlation coefficient \\
sigm($x$) & Sigmoid (logistic) function,$\dfrac{1}{1+e^{-x}}$\\
$\sigma^2$ & Variance\\
$\Sigma$ & Covariance matrix\\
var[$x$] & Variance of $x$\\
$\nu$ & Degrees of freedom parameter\\
Z & Normalization constant of a probability distribution\\
\noalign{\smallskip}\hline\noalign{\smallskip}
\end{longtable}

\section*{Machine learning/statistics notation}
In general, we use upper case letters to denote constants, such as $C, K, M, N, T$, etc. We use lower case letters as dummy indexes of the appropriate range, such as $c=1:C$ to index classes, $i=1:M$ to index data cases, $j=1:N$ to index input features, $k=1:K$ to index states or clusters, $t=1:T$ to index time, etc.

We use $x$ to represent an observed data vector. In a supervised problem, we use $y$ or $\vec{y}$ to represent the desired output label. We use $\vec{z}$ to represent a hidden variable. Sometimes we also use $q$ to represent a hidden discrete variable.

\begin{longtable}{p{2.4cm}p{8.9cm}}
Symbol & Meaning \\
\noalign{\smallskip}\hline\noalign{\smallskip}
$C$ & Number of classes\\
$D$ & Dimensionality of data vector (number of features)\\
$N$ & Number of data cases\\
$N_c$ & Number of examples of class $c$,$N_c=\sum_{i=1}^{N}\mathbb{I}(y_i=c)$\\
$R$ & Number of outputs (response variables)\\
$\mathcal{D}$ & Training data $\mathcal{D}=\left\{(\vec{x}_i,y_i) | i=1:N\right\}$\\
$\mathcal{D}_{test}$ & Test data\\
$\mathcal{X}$ & Input space\\
$\mathcal{Y}$ & Output space\\
$K$ & Number of states or dimensions of a variable (often latent)\\
$k(x,y)$ & Kernel function\\
$\vec{K}$ & Kernel matrix\\
$\mathcal{H}$ & Hypothesis space\\
$L$ & Loss function \\
$J(\vec{\theta})$ & Cost function\\
$f(\vec{x})$ & Decision function\\
$P(y|\vec{x})$ & TODO\\
$\lambda$ & Strength of $\ell_2$ or $\ell_1 regularizer$\\
$\phi(x)$ & Basis function expansion of feature vector $\vec{x}$\\
$\Phi$ & Basis function expansion of design matrix $\vec{X}$\\
$q()$ & Approximate or proposal distribution\\
$Q(\vec{\theta},\vec{\theta}_{old})$ & Auxiliary function in EM\\
$T$ & Length of a sequence\\
$T(\mathcal{D})$ & Test statistic for data\\
$\vec{T}$ & Transition matrix of Markov chain\\
$\vec{\theta}$ & Parameter vector\\
$\vec{\theta}^{(s)}$ & $s$'th sample of parameter vector\\
$\hat{\vec{\theta}}$ & Estimate (usually MLE or MAP) of $\vec{\theta}$\\
$\hat{\vec{\theta}}_{MLE}$ & Maximum likelihood estimate of $\vec{\theta}$\\
$\hat{\vec{\theta}}_{MAP}$ & MAP estimate of $\vec{\theta}$\\
$\bar{\vec{\theta}}$ & Estimate (usually posterior mean) of  $\vec{\theta}$\\
$\vec{w}$ & Vector of regression weights (called $\vec{\beta}$ in statistics)\\
b & intercept (called $\varepsilon$ in statistics)\\
$\vec{W}$ & Matrix of regression weights\\
$x_{ij}$ & Component (i.e., feature) $j$ of data case $i$ ,for $i=1:N ,j=1:D$\\
$\vec{x}_i$ & Training case, $i=1:N$\\
$\vec{X}$ & Design matrix of size $N \times D$\\
$\bar{\vec{x}}$ & Empirical mean $\bar{\vec{x}}=\dfrac{1}{N}\sum_{i=1}^{N} \vec{x}_i$\\
$\tilde{\vec{x}}$ & Future test case\\
$\vec{x}_*$ & Feature test case\\
$\vec{y}$ & Vector of all training labels $\vec{y} =(y_1,...,y_N)$\\
$z_{ij}$ & Latent component $j$ for case $i$\\
\noalign{\smallskip}\hline\noalign{\smallskip}
\end{longtable}
\mainmatter%%%%%%%%%%%%%%%%%%%%%%%%%%%%%%%%%%%%%%%%%%%%%%%%%%%%%%%
该神仙,知何途;混样近,连续组\\
问问这个神仙,知不知道哪条路径是对的,妖怪(混子)走过来,
连连阻止前进
\begin{enumerate}
\item 该:概率论
\item 神:神经网络
\item 仙:线性回归和线性分类,广义线性模型
\item 知:支持向量机
\item 何:核函数
\item 途:图模型
\item 混:混合模型
\item 样:抽样
\item 近:近似推断
\item 连:连续隐变量
\item 续:序列化数据
\item 组:组合模型
\end{enumerate}
\chapter{Introduction}

\chapter{Probability Distribution}
\label{probability_distribution}

\section{高斯分布}
\begin{equation}
N(x|\mu, \Sigma), \Lambda \equiv \Sigma^{-1}
\end{equation}

\begin{equation}
\begin{aligned}
x = \begin{pmatrix}
 x_a\\ 
 x_b
  \end{pmatrix},
\mu = 
\begin{pmatrix}
 \mu_a \\ 
 \mu_b
  \end{pmatrix}
\end{aligned}
\end{equation}

\begin{equation}
\begin{aligned}
\Sigma = \begin{pmatrix}
\Sigma_{aa}&\Sigma_{ab}\\
\Sigma_{ba}&\Sigma_{bb}
\end{pmatrix},
\Lambda = 
\begin{pmatrix}
\Lambda_{aa}& \Lambda_{ab}\\
\Lambda_{ba}& \Lambda_{bb}
\end{pmatrix}
\end{aligned}
\end{equation}

\subsection{条件分布}
\begin{equation}
\begin{aligned}
p(x_a|x_b) = N(x_a|\mu_{a|b}, \Lambda_{aa}^{-1})
\\
\mu_{a|b} = \mu_a - \Lambda_{aa}^{-1}\Lambda_{ab}(x_b - \mu_b)
\end{aligned}
\end{equation}

\subsection{边缘分布}
\begin{equation}
\begin{aligned}
p(x_a) = N(x_a|\mu_a, \Sigma_{aa})
\end{aligned}
\end{equation}

\section{贝叶斯推理}
\begin{equation}
p(x) = N(x|\mu, \Lambda^{-1})
\end{equation}

\begin{equation}
p(y|x) = N(y|Ax + b, L^{-1})
\end{equation}

\begin{equation}
p(y) = N(y|A\mu + b, L^{-1} + A\Lambda^{-1}A^T)
\end{equation}

\begin{equation}
p(x|y) = N(x|\Sigma\{A^TL(y-b) + \Lambda\mu\}, \Sigma)
\end{equation}

\begin{equation}
\Sigma = (\Lambda + A^TLA)^{-1}
\end{equation}

\subsection{伽玛分布}
精度的符号问题,$\alpha$是一种常用记号,在线性回归里面
经常使用$\alpha,\beta$,使用$\lambda$也很容易理解,在线性代数里面
$\lambda$经常用表示特征值,协方差矩阵用SAS分解以后,就是一个对角
阵,每个元素就是特征值,这也就很容易理解为什么精度矩阵用$\Lambda$了,
因为它本就是特征值的矩阵。
伽玛分布常作为高斯分布的共轭出现。

\begin{equation}
Ga(\lambda|a, b) = \frac{b^a}{\Gamma(a)}\lambda^{a-1}e^{-b\lambda}
= z \lambda^{a-1}e^{-b\lambda}
\end{equation}

\begin{equation}
p(x|\lambda) = \prod_{n=1}^NN(x_n|\mu, \lambda^{-1})
= z \lambda^{\frac{N}{2}}e^{-\frac{\lambda}{2}\sum_{n=1}^N(x_n-\mu)^2}
= z \lambda^ke^{-l \lambda}
\end{equation}

\subsection{高斯伽玛分布}
\begin{equation}
p(\mu, \lambda) = p(\mu|\lambda)p(\lambda)
= N(\mu|\mu_0, (\beta \lambda)^{-1})Ga(\lambda|a, b)
\end{equation}

\subsection{指数族}

\begin{enumerate}
\item 指数族
\begin{equation}
p(x|\eta) = h(x)g(\eta)exp\{\eta^Tu(x)\}
\end{equation}
\end{enumerate}

\subsection{极大似然估计和充分统计}

\begin{enumerate}
\item 似然函数
\begin{equation}
p(X|\eta) = (\prod_{n=1}^Nh(x_n))g(\eta)^N \exp \{
\eta^T \sum_{n=1}^N u(x_n)
\}
\end{equation}
\item 充分统计
\begin{equation}
-\nabla\ln(\eta_{ML}) = \frac{1}{N}\sum_{n=1}^N{u(x_n)}
\end{equation}
\end{enumerate}

\subsection{共轭先验}
\begin{equation}
p(\eta|x, v) = f(x, v)g(\eta)^v\exp\{v\eta^Tx\}
\end{equation}


\subsection{问题}
\begin{enumerate}
\item 如何根据2.228的定义求出$\eta_{ML}$?\\
根据公式2.203, 2.217以及2.223可以知道,$g(\eta)$与均值与方差的关系,
因此只要我们知道了均值
与方差就可以知道$\eta_{ML}$。$\ln g(\eta)$称作配分函数,
配分函数的一个很好性质就是它的导数就$E(u(x))$,二次导
数是$var(u(x))$,因此由2.228可以得到$\eta$。
\end{enumerate}

\chapter{Linear models and regression}

\chapter{Linear models for classification}

\section{判别式模型}
\subsection{概率回归}
\begin{enumerate}
\item 激活函数
\begin{equation}
f(a) = E(a) = \int_{-\infty}^ap(\theta)d\theta
\end{equation}
\item 概率函数
\begin{equation}
\Phi(a) = \int_{-\infty}^a
f(a) = E(a) = \int_{-\infty}^aN(\theta|0, 1)d\theta)
\end{equation}
\end{enumerate}
\subsection{标准链接函数}
\section{问题}
公式4.146

\chapter{Neural networks}

\chapter{Kernel Methods}
\section{问题}
\begin{enumerate}
\item 什么是核?
\item 什么是高斯过程?
\item 为什么使用高斯过程?
\item P292 固定核?
\item 对偶形式的表达,出现了参数个数的变化,由$|w|$变为
$|a|$,是否会过拟合,能带来哪些好处,特征多的影响,特征少的
影响。例如文档分类,特征多,训练集少。
\end{enumerate}
\section{核}
\begin{enumerate}
\item 高斯核
\begin{equation}
k(x_m, x_n) = exp(-||x_m - x_n||^2/2\sigma^2)
\end{equation}
\item 指数核
\begin{equation}
k(x_m, x_n) = exp(-\theta|x_m - x_n|)
\end{equation}
\end{enumerate}

\section{对偶表示}
\subsection{线性回归}
\begin{equation}
y(x) = w^T\phi(x)
\end{equation}
\subsection{对偶表示}
\begin{equation}
y(x) = a^T\Phi\phi(x) = a^Tk(x)
\end{equation}
\subsection{二者之间的关系}
\begin{equation}
J(a) = \frac{1}{2}\sum_{n=1}^N(k(x_n)^Ta - t_n)^2
= \frac{1}{2}||\Phi\Phi^Ta - t||^2
\end{equation}

\begin{equation}
J(w) = \frac{1}{2}\sum_{n=1}^N(\phi(x_n)^Tw - t_n)^2
= \frac{1}{2}||\Phi w - t||^2
\end{equation}

\subsection{协方差矩阵}

\begin{equation}
cov(y) = E(yy^T) = E(\Phi ww^T\Phi^T)
= \Phi E(ww^T) \Phi^T
= \Phi cov(w)\Phi^T
\end{equation}

\begin{equation}
C_{N+1} = \begin{pmatrix}
C_N & k\\
k^T & c
\end{pmatrix}
\end{equation}

对偶形式的优点是,不用显示的写出特征空间向量,只关心
$k(x_m, x_n)$, 核函数完成了由特征空间到核空间的转化。
最重要的是,可以使用不同的核函数进行替换。

\section{高斯过程}
\begin{enumerate}
\item 高斯过程是定义在y上联合分布服从高斯分布的概率分布,
简单说就是高斯过程就是
定义在y上的高斯分布。高斯过程可以由完全由二次统计决定,即均值和方差。
根据高斯分布的性质,高斯分布的边缘分布和条件分布都是高斯分布。
y的取值范围是R,因此是一个无穷维。在实际应用中,只考虑训练集大小
的维度。
回归问题的目标是$p(y_{N+1}) = N(y_{N+1}|\mu_{N+1}, \sigma_{N+1})$。
因此用高斯过程解决回归问题的关键是如何求出均值和方差。
高斯过程中的协方差$E[y(x_n)y(x_m)]$可以用核的方式给出来。

定一个实矩阵 A,矩阵 $A^TA$ 是 A 的列向量的格拉姆矩阵,
而矩阵$AA^T$ 是 A 的行向量的格拉姆矩阵。
\item 高斯过程与核函数\\
高斯过程协方差矩阵自然的与核函数联系在一起。

\end{enumerate}
\section{Summary}
\begin{enumerate}
\item 高斯过程与线性回归之间的对偶
\item 高斯过程与隐变量,对于输出y,y和y之间
实际上是有关系的,通过指定条件w(相当于一个隐变量),
从而使得y只依赖于w和x,与产生式模型的比较。
\item 高斯过程与线性模型\\
线性模型用w表示了x和y之间的关系,并且通过w之间的
协方差表示y之间的协方差。高斯过程抛弃了w,直接表示
y之间的关系,这种关系通过核表示出来。高斯过程用核$k(x_m,x_n)$
取代了线性模型中的w。
\item 回归与分类,回归问题+一个激活函数,转变为分类问题
\end{enumerate}



\chapter{Sparse kernel machines}
\section{问题}
\begin{enumerate}
\item 当数据量很大的时候,计算所有核函数对很复杂,
如何只利用局部核解决问题? 即如何通过少量核求解问题?
\item 什么是无穷维?
\item 对偶表示的好处?
\item 为什么提出svm?
\item svm为什么需要正定核?
\item rvm、type2极大似然估计、高斯过程的关系?
\end{enumerate}

\section{最大间隔分类器}
\subsection{损失函数}
\begin{equation}
L(w) = \sum_{n=1}^N E_\infty(y(x_n)t_n - 1) + \lambda||w||^2
\end{equation}
\section{重叠分类分布}

\subsection{损失函数}
\begin{equation}
L(w) = C\sum_{n=1}^N\xi_n + \frac{1}{2}||w||^2
\end{equation}

\chapter{Graphical models}
\section{简介}
图模型提供了一个工具,描述问题,模拟问题,推理问题\cite{longxing2012machinelearning}.
三个基本问题,
\begin{enumerate}
\item 有了问题,如何模拟,怎么描述,把自己的先验知识放到问题里面; 
\item 推理,有了模型, 可以去回答什么问题;
\item 学习,参数不了解,结构化的东西不了解,通知数据学习参数和结构;
\end{enumerate}
用紧凑的方式表示变量之间的关系。
\begin{enumerate}
\item 什么样的图模型更适合一个问题?
\item 对定一个图模型,求P(x|y)
\item 如何得到图模型的结构和参数
\end{enumerate}

\section{贝叶斯网}
\subsection{有向分离}
\begin{enumerate}
\item 高斯分布
\begin{equation}
p(D|\mu) = \prod_{n=1}^Np(x_n|\mu)
\end{equation}

\begin{equation}
p(D) \neq \prod_{n=1}^Np(x_n)
\end{equation}

\item 朴素贝叶斯
\begin{equation}
p(x|z) = \prod_{m=1}^M(x_m|z)
\end{equation}
\end{enumerate}

\subsubsection{有向图过滤器}
有向分离是判断贝叶斯网中两个节点是否条件独立的方法。
图的表达能力,越连通表达能力越强,越独立表达能力越弱。
全联全概率或者全连通图表达能力最强,可以表达一切分布,
因此可以让所有分布都通过。对于一个完全分解的图,表达能力最弱。
\section{概念}
\begin{enumerate}
\item 条件独立性
\item 有向分离
\end{enumerate}

\chapter{Mixture models and EM}

\chapter{Approximate inference}

动机,准确推理不可行,转化为用一些可行的分布去近似那些不可
行的分布, 在保证可行的情况下,尽可能提供丰富的分布,从中找出
最优的。
\section{变分推理}
\begin{enumerate}
\item 变化推理的目标,后验分布,似然函数
\begin{equation}
p(x)
\end{equation}
\begin{equation}
p(z|x)
\end{equation}

\item 对数似然等于联合相对熵减去条件相对熵(对数似然分解)
\begin{equation}
\ln p(x) = L(q) + KL(q||p)
\end{equation}
\begin{equation}
L(q) = KL(q||p(x,z))
\end{equation}
\begin{equation}
KL(q||p) = KL(q||p(x|z))
\end{equation}
与以往不同,这里没有出现参数$\theta$,这是因为在这里所有的参数也
看作是随机变量,因此Z已经包含了参数。
\end{enumerate}

\section{因式分布}

\begin{equation}
q(z) = \prod_{m=1}^Mq_m(z_m)
\end{equation}
参照模图型分解
\subsection{最优解}
因式最优解等于非因式联合期望
\begin{equation}
\ln q_j = E_{-j}[\ln p(x, z)] + const
\end{equation}

\section{问题}
\begin{enumerate}
\item 为什么使用近似推理?
\item 什么是变分?
\item 变分的关键地方是什么?
\item EM算法用到了哪些技巧?是否可以重用?
\item 分离技巧,RVM,变分都有体现。
\end{enumerate}


\chapter{Sampling methods}

\chapter{Continuous latent variables}
\section{问题}
\begin{enumerate}
\item PPCA与产生式模型
\item 从自由度理解PCA
\item PCA、PPCA和FA\\
PCA强调数据空间到隐空间,PPCA和FA强调隐空间到数据空间
\end{enumerate}

\chapter{Sequential data}

\chapter{Combining models}

\chapter{Question}
\section{问题}

\begin{enumerate}
\item PCA降维与隐变量降维之间的关系?\\
\item 矩阵稀疏什么时候好,什么时候坏?\\
一般稀疏的矩阵容易处理,网络社团算法一般都是在
稀疏矩阵中做的。对于推荐问题,稀疏以后就面临过拟合,推荐不准确的困难。
\item 似然函数的写法?\\
回归问题、产生式模型、判别式模型
\begin{equation}
p(D|w) = \prod^N_{n=1} p(y_n|w^Tx_n)
\end{equation}
分类问题
\begin{equation}
p(D|\theta) = \prod_{n=1}^N p(x_n, y_n|\theta)
= \prod_{n=1}^Np(x_n|\theta)p(y_n)
\end{equation}
\item 高斯过程图模型?
\item EM算法,广义EM算法,非参方法EM算法?
\item 是否可以把分类问题的标签去掉变成聚类问题?
\end{enumerate}

\section{EM算法}
\begin{enumerate}
\item EM算法是否可以用于没有隐变量的情况?
\item 可不可以把隐变量当成参数使用梯度下降算法?
\item FA可以使用梯度下降算法吗?
\end{enumerate}
\section{Idea}
\begin{enumerate}
\item 熟悉的内容讲公式
\item 不熟悉的内容讲直观
\item 组成一个加强组 
\end{enumerate}

\section{目标}
\begin{enumerate}
\item 发高质量论文
\item 至少精读一本书(PRML、MLAPP)
\item 集中阅读文献(略读)
\end{enumerate}




\backmatter%%%%%%%%%%%%%%%%%%%%%%%%%%%%%%%%%%%%%%%%%%%%%%%%%%%%%%%
\appendix
%\include{appendix}
\include{glossary}
\printindex
\clearpage
\end{CJK}

\bibliographystyle{plain}
\bibliography{mybib}

\end{document}
