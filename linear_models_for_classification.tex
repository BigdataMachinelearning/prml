\chapter{Linear Models for Classification}
\section{输入变量是离散型}
\subsection{联合分布}
\begin{enumerate}
\item y服从伯努利分布(二分类)
\begin{equation}
\begin{aligned}
p(x, y) = p(y)p(x|y) = \pi^y(1-\pi)^{1-y}
p(x|\mu_0)^{1-y}p(x|\mu_1)^y
= [(1-\pi)p(x|\mu_0)]^{1-y}[\pi p(x|\mu_1)]^y
\end{aligned}
\end{equation}
\begin{enumerate}
\item x服从伯努利分布
\begin{equation}
p(x, y) =[(1-\pi)\mu_0^x(1-\mu_0)^{1-x}]^{1-y}
[\pi\mu_1^x(1-\mu_1)^{1-x}]^y
\end{equation}
\item x服从离散分布
\begin{equation}
p(x, y) =[(1-\pi)\prod_{k=1}^K\mu_{0k}^{x_k}]^{1-y}
[\pi \prod_{k=1}^K\mu_{1k}^{x_k}]^y
\end{equation}
\end{enumerate}

\item y服从离散分布(多分类)
\begin{equation}
p(x, y) = p(y)p(x|y) = \prod_{k=1}^K\pi_k^{y_k}
\prod_{k=1}^Kp(x|\theta_k)^{y_k}
=\prod_{k=1}^K[\pi_kp(x|\theta_k)]^{y_k}
\end{equation}
\begin{enumerate}
\item x服从伯努利分布
\begin{equation}
\begin{aligned}
p(x, y) =\prod_{k=1}^K[\pi_k\mu_k^x(1-\mu_k)^{1-x}]^{y_k}
\end{aligned}
\end{equation}

\item x服从离散分布
\begin{equation}
\begin{aligned}
p(x, y) =\prod_{c=1}^C[\pi_k\prod_{k=1}^K\mu_{ck}^{x_k}]^{y_c}
\end{aligned}
\end{equation}

\end{enumerate}
\end{enumerate}

\subsection{极大似然估计}
\begin{enumerate}
\item $y\sim B(y|\pi), x\sim B(x|\mu)$
\begin{equation}
logp(D) = \sum_{n=1}^N
\{(1-y)[log(1-\pi) + xlog\mu_0 + (1-x)log(1-\mu_0)]
+ y[log\pi + xlog\mu_1 + (1-x)log(1-\mu_1)]\}
\end{equation}

令
\begin{equation}
\frac{\partial logp(D)}{\partial \pi}
= \frac{1}{\pi}\sum_{n=1}^Ny - \frac{1}{1-\pi}\sum_{n=1}^N(1-y)
=0
\end{equation}
得
\begin{equation}
\pi = \frac{\sum y}{N}
\end{equation}

令
\begin{equation}
\frac{\partial logp(D)}{\partial \mu_1}
= \frac{1}{\mu_1}\sum_{n=1}^Nyx - \frac{1}{1-\mu_1}\sum_{n=1}^Ny(1-x)
=0
\end{equation}
得
\begin{equation}
\mu_1 = \frac{\sum xy}{\sum y}
\end{equation}

令
\begin{equation}
\frac{\partial logp(D)}{\partial \mu_0}
= \frac{1}{\mu_0}\sum_{n=1}^N(1-y)x - \frac{1}{1-\mu_0}\sum_{n=1}^N(1-y)(1-x)
=0
\end{equation}
得
\begin{equation}
\mu_0 = \frac{\sum x(1-y)}{\sum (1-y)}
\end{equation}

\item $y\sim Cat(y|\pi), x\sim Cat(x|\mu)$
\begin{equation}
\begin{aligned}
logp(D) =\sum_{n=1}^N\sum_{c=1}^Cy_{nc}
[log\pi_k + \sum_{k=1}^Kx_{nk}log\mu_{ck}]
\end{aligned}
\end{equation}
另外
\begin{equation}
\sum_{k=1}^K\mu_{ck} = 1
\end{equation}
最大化下面的式子
\begin{equation}
\begin{aligned}
\sum_{n=1}^N\sum_{c=1}^Cy_{nc}[log\pi_k + \sum_{k=1}^Kx_{nk}log\mu_{ck}]
+ \lambda (\sum_{k=1}^K\mu_{ck} - 1)
\end{aligned}
\end{equation}
得
\begin{equation}
\mu_{ck} = \frac{\sum_{n=1}^N y_{nc}x_{nk}}{\sum_{n=1}^N\sum_{k=1}^K
y_{nc}x_{nk}}
= \frac{N_{ck}} {N_c}
\end{equation}


\end{enumerate}

\section{朴素贝叶斯}
似然函数
\begin{equation}
p(D|\pi, \theta) = \prod_{n=1}^N\{
(\prod_{c=1}^C\pi_c^{y_c^n})
(\prod_{m=1}^M\prod_{c=1}^Cp(x_m^n|\theta_{mc})^{y_c^n}
\}
\end{equation}

对数似然函数
\begin{equation}
logp(D|\pi, \theta) = \sum_{c=1}^CN_clog\pi_c
+ \sum_{m=1}^M\sum_{c=1}^C\sum_{n=1}^Ny_c^nlogp(x_m^n|\theta_{mc})
\end{equation}
用极大似然估计, $\pi_c = \frac{N_c}{N}$

以上是mlapp中给出的公式,太复杂,不够简练,简练的公式容易让人记忆,
容易提炼出本质。下面给出一种紧凑的格式。

似然函数
\begin{equation}
p(D|\pi, \theta) = \prod_{c,n}
[\pi_c\prod_mp(x_m^n|\theta_{mc})]^{y_c^n}
\end{equation}

对数似然函数
\begin{equation}
logp(D|\pi, \theta) = \sum_{c, n}y_c^n
[log\pi_c + \sum_mlogp(x_m^n|\theta_{mc})]
\end{equation}
\subsection{贝努力分布}

\begin{equation}
p(x_m^n|\theta_{mc}) = p(x_m^n|\mu_{mc})
= \mu_{mc}^{x_m^n}(1 - \mu_{mc})^{1 - x_m^n}
\end{equation}

对数似然函数

\begin{equation}
\begin{aligned}
logp(D|\pi, \theta) = \sum_{c,n}
y_c^n\{log\mu_c + \sum_m[x_m^nlog\mu_{mc} + (1 - x_m^n)
log(1 - \mu_{mc})]\}
\\= \sum_c\{
N_clog\pi_c + \sum_m[N_{mc}log\mu_{mc} 
+ (N_c - N_{mc})log(1-\mu_{mc})]
\}
\end{aligned}
\end{equation}

用极大似然估计

\begin{equation}
\frac{\partial logp(D|\pi, \theta)} {\partial \mu_{mc}}
= \mu^{-1}_{mc}N_{mc} - (1 - \mu_{mc})^{-1}(N_c - N_{mc})
\end{equation}

令
\begin{equation}
\frac{\partial logp(D|\pi, \theta)} {\partial \mu_{mc}}
= 0
\end{equation}
得
$\mu_{mc} = \frac{N_{mc}}{N_c}$

\subsection{离散分布}
\begin{equation}
Cat(x_m^n|\mu_{mc}) = \prod_{k = 1}^K\mu_{mck}^{x_{mk}^n}
\end{equation}

离散分布似然函数
\begin{equation}
p(D|\pi, \theta) = \prod_{n=1}^N\{
(\prod_{c=1}^C\pi_c^{y_c^n})
(\prod_{m=1}^M\prod_{c=1}^C
\{\prod_{k = 1}^K\mu_{mck}^{x_{mk}^n}
\}^{y_{c}^n}
\}
\end{equation}
写成简洁格式如下:
\begin{equation}
p(D|\pi, \theta) = \prod_{c, n}
(\pi_c\prod_{m, k}\mu_{mck}^{x_{mk}^n})^{y_c^n}
\end{equation}
相应的对数似然函数
\begin{equation}
\begin{aligned}
logp(D|\pi, \theta) = \sum_{c, n}y_c^n[log\pi_c + \sum_{m, k}x_{mk}^nlog\mu_{mck}]
\\= \sum_c[N_clog\pi_c + \sum_{m, k}log\mu_{mck}\sum_ny_c^nx_{mk}^n]
\\=\sum_c[N_clog\pi_c + \sum_{m,k}N_{mck}log\mu_{mck}]
\end{aligned}
\end{equation}
注意到
\begin{equation}
\sum_{m,k} \mu_{mck} = 1
\end{equation}
因此,可以通过最大化下面的式子求解:
\begin{equation}
\sum_{m,c,k}N_{mck}log\mu_{mck} +\lambda(\sum_{m,k}\mu_{mck} -1)
\end{equation}
求导后,可得:
\begin{equation}
\mu_{mck} = -\frac{N_{mck}}{\lambda}
\end{equation}
根据条件概率定义:$\sum_{m,k}\mu_{mck} = 1$, 而不是$\sum_{m,c,k}\mu_{mck} = 1$, 
得$\mu_{mck} = \frac{N_{mck}} {N_c}$
从以上可以看出,朴素贝叶斯提供了一个框架,针对不同特征的具体分布,
代入基本公式可以用极大似然求出参数。

\section{判别式模型}
\subsection{概率回归}
\begin{enumerate}
\item 激活函数
\begin{equation}
f(a) = E(a) = \int_{-\infty}^ap(\theta)d\theta
\end{equation}
\item 概率函数
\begin{equation}
\Phi(a) = \int_{-\infty}^a
f(a) = E(a) = \int_{-\infty}^aN(\theta|0, 1)d\theta)
\end{equation}
\end{enumerate}
\subsection{标准链接函数}
\section{问题}
\begin{enumerate}
\item 公式4.146
\item kernel logistiction regression
\end{enumerate}
