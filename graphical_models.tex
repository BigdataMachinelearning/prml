\chapter{Graphical models}
\section{简介}
图模型提供了一个工具,描述问题,模拟问题,推理问题\cite{longxing2012machinelearning}.
三个基本问题,
\begin{enumerate}
\item 有了问题,如何模拟,怎么描述,把自己的先验知识放到问题里面; 
\item 推理,有了模型, 可以去回答什么问题;
\item 学习,参数不了解,结构化的东西不了解,通知数据学习参数和结构;
\end{enumerate}
用紧凑的方式表示变量之间的关系。
\begin{enumerate}
\item 什么样的图模型更适合一个问题?
\item 对定一个图模型,求P(x|y)
\item 如何得到图模型的结构和参数
\end{enumerate}

\section{贝叶斯网}
\subsection{有向分离}
\begin{enumerate}
\item 高斯分布
\begin{equation}
p(D|\mu) = \prod_{n=1}^Np(x_n|\mu)
\end{equation}

\begin{equation}
p(D) \neq \prod_{n=1}^Np(x_n)
\end{equation}

\item 朴素贝叶斯
\begin{equation}
p(x|z) = \prod_{m=1}^M(x_m|z)
\end{equation}
\end{enumerate}

\subsubsection{有向图过滤器}
有向分离是判断贝叶斯网中两个节点是否条件独立的方法。
图的表达能力,越连通表达能力越强,越独立表达能力越弱。
全联全概率或者全连通图表达能力最强,可以表达一切分布,
因此可以让所有分布都通过。对于一个完全分解的图,表达能力最弱。
\section{概念}
\begin{enumerate}
\item 条件独立性
\item 有向分离
\end{enumerate}
