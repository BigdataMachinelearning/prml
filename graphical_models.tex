\chapter{Graphical models}
\section{简介}
图模型提供了一个工具,描述问题,模拟问题,推理问题\cite{longxing2012machinelearning}.
三个基本问题,
\begin{enumerate}
\item 有了问题,如何模拟,怎么描述,把自己的先验知识放到问题里面; 
\item 推理,有了模型, 可以去回答什么问题;
\item 学习,参数不了解,结构化的东西不了解,通知数据学习参数和结构;
\end{enumerate}
用紧凑的方式表示变量之间的关系。
\begin{enumerate}
\item 什么样的图模型更适合一个问题?
\item 对定一个图模型,求P(x|y)
\item 如何得到图模型的结构和参数
\end{enumerate}

\section{条件独立性}
给定一个联合分布,我们关心哪些变量之间是条件独立的。或者在给定条件下判断两个
变量是否独立。

\begin{equation}
A\bot B | C
\end{equation}
\section{贝叶斯网}
\subsection{有向图过滤器}
\begin{enumerate}
\item 有向图有什么用?\\
更方便的表达独立性,一个有向图对应着一个联合分布的一个
形式,我们关心的是该有向图表达了什么样的独立性,
即在给定什么样的条件下两个变量是独立的。
独立性表达能力是一个图的本质属性。如果两个图的独立性表达能力
相同,可以认为是等价的。
一个有向图最直观的分解方式就是父节点分解法。
条件独立性的重要性在于把本不独立的两个事件看作是独立的,例如朴素
贝叶斯。
如果需要的条件越少,那么可以认为独立性越强,相反信赖性越强。
\item 有向分离理论,(过滤器原理)\\
给定一个有向图,可以直接写出一个分解形式,每个因式是一个条件分布;
同样,按照有向分离原则,也可以得到一种条件独立性;
有向分离理论告诉我们这两种形式的条件独立性是等价的。
这种等性的意义在于用一个可以直观容易表达的公式表达了一种复杂的需求。
因式分解是数学表达,条件独立是需求。\\
把一个有向图看作一个过滤器,第一种分布是让满足图分解的分布通过,
第二种是让满足有向分立的分布通过,有向分离理论告诉我们这两种分布集合
等价。即一个图的分解分布和有向分立分布在独立性上等价的。\\
两个极限,全连通图和全分解分布,全连通图可以让任意一个分布通过?全分解
分布可以通过任意图。哪果一个分布可以被有向图表达,那么它一定可以通过
全连通图。
因为有些分布有向图是无法表达的,是否意味着有些分布
不能通过全连通图。
\end{enumerate}

有向分离是判断贝叶斯网中两个节点是否条件独立的方法。
图的表达能力,越连通表达能力越强,越独立表达能力越弱。
全联全概率或者全连通图表达能力最强,可以表达一切分布,
因此可以让所有分布都通过。对于一个完全分解的图,表达能力最弱。

\subsection{有向分离}
\begin{enumerate}
\item 高斯分布
\begin{equation}
p(D|\mu) = \prod_{n=1}^Np(x_n|\mu)
\end{equation}

\begin{equation}
p(D) \neq \prod_{n=1}^Np(x_n)
\end{equation}

\item 朴素贝叶斯
\begin{equation}
p(x|z) = \prod_{m=1}^M(x_m|z)
\end{equation}
\end{enumerate}


\section{概念}
\begin{enumerate}
\item 条件独立性
\item 有向分离
\end{enumerate}
