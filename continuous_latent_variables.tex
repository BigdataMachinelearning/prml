\chapter{Continuous latent variables}

\section{PCA}
\begin{equation}
J(W,Z) = \frac{1}{N}\sum_{n=1}^N||x_i - Wz_i||^2
\end{equation}
\begin{equation}
J(W,Z) = ||\Phi - ZW^T||^2_F
\end{equation}
佛罗贝尼乌斯范数(Frobenius norm)
\begin{equation}
||A||_F = \sqrt{\sum_{n=1}^N\sum_{m=1}^Ma_{nm}^{2}} = \sqrt{tr(A^TA)}
 = ||A(:)||_2
\end{equation}

\subsection{最大化方差}
\begin{equation}
var(ux) = \frac{1}{N}
\sum_{n=1}^N\{u^Tx_n - u^T\bar{x}\}^2 = u^TSu
\end{equation}

\begin{equation}
u^Tu = 1
\end{equation}

\begin{equation}
L(u) = u^TSu + \lambda(1 - u^Tu)
\end{equation}

\begin{equation}
Su = \lambda u
\end{equation}

\begin{equation}
var(ux) = u^TSu = \lambda
\end{equation}

\subsection{PCA算法}
\begin{enumerate}
\item 协方差矩阵
\begin{equation}
 \Sigma = \frac{1}{N}\sum_{n=1}^N(x_nx_n^T)^2
\end{equation}

\item SVD分解
\begin{equation}
 \Sigma = [U, S, V]
\end{equation}

\begin{equation}
W = U_{1:k}
\end{equation}

\item数据压缩
\begin{equation}
 z = W^Tx
\end{equation}

\begin{equation}
 Z = XW 
\end{equation}
\end{enumerate}


\subsection{维数选取}

\begin{enumerate}
\item 均方映射误差
\begin{equation}
 E(c) = \frac{1}{N}\sum_{n=1}^N(x_n - x_{app})^2
\end{equation}
\item 数据方差

\begin{equation}
 Cov(d) = \frac{1}{N}\sum_{n=1}^Nx_n^2
\end{equation}

\item 判断标准
\begin{equation}
\frac{E(c)}{Cov(d)} \leq \theta_0
\end{equation}

\item 特征值判断
\begin{equation}
\Sigma = [U, S, V]
\end{equation}

\begin{equation}
\frac{\sum_{i = 1}^k S_{ii}}{\sum_{i = 1}^M S_{ii}} \leq \theta_0
\end{equation}
\end{enumerate}

\subsection{数据解压}

\subsection{PCA应用}
\begin{enumerate}
\item 数据压缩和存储
\item 数据可视化
\item 加快学习\\
只能在训练集上进行压缩,在原数据上进行测试
\item 不可以用来避免过拟合
\end{enumerate}
本节内容来源\cite{andrew2013courseramachinelearning}
\section{PPCA}
\section{FA}

\section{问题}
\begin{enumerate}
\item PPCA与产生式模型
\item 从自由度理解PCA
\item PCA、PPCA和FA\\
PCA强调数据空间到隐空间,PPCA和FA强调隐空间到数据空间
\item PCA与fisher算法之间是什么关系?(PCA无监督、fisher有监督)
\item PCA与岭回归
\item PCA的解是否唯一,当最大化方差的时候,旋转主成份的基,解不变。
\item PCA与SVD的关系
\item PCA如何解压缩?如何求解压矩阵
\end{enumerate}
\subsection{PCA与SVD的关系}
PCA强调对数据进行降维,这种降维方式可以通过协方差矩阵的
特征值和特征向量得到。SVD是对数据进行矩阵分解,得到$USV^T$。
两者侧重点不同,一是降维,一是分解。二者联系很大,
因为它们都与特征值(奇异值有关)。
\subsection{FA中W的不可判别性}
用最大似然估计出现$WW^T$,因此W不在唯一。这是否说明具体的基与
问题是透明的,为什么需要W呢,我们只需要$WW^T$可以解决问题吗?
在支持向量机中,通过对偶的形势$x^Tx$的形势,从而可以用核
隐性的升维。这里和核又有什么联系呢?在什么场景中可以只用到
$WW^T$就可以了。

\subsubsection{场景一 低维度分类}
低维度分类$z = W^Tx$,因此在低维度分类的时候用到如下核
$z^Tz = x^TWW^Tx$,这时候就不再需要W。 

