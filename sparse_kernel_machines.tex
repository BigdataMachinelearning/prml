\chapter{Sparse kernel machines}
\section{问题}
\begin{enumerate}
\item 当数据量很大的时候,计算所有核函数对很复杂,
如何只利用局部核解决问题? 即如何通过少量核求解问题?
\item 什么是无穷维?
\item 对偶表示的好处?
\item 为什么提出svm?
\item svm为什么需要正定核?
\item svm、rvm、type2极大似然估计、高斯过程的关系?
\item 对偶、非参方法、核?
\item 统计学习方法原始问题和对偶问题的解的大小?
\end{enumerate}

\section{最大间隔分类器}
\subsection{损失函数}
\begin{equation}
L(w) = \sum_{n=1}^N E_\infty(y(x_n)t_n - 1) + \lambda||w||^2
\end{equation}
\section{重叠分类分布}

\subsection{损失函数}
\begin{equation}
L(w) = C\sum_{n=1}^N\xi_n + \frac{1}{2}||w||^2
\end{equation}
