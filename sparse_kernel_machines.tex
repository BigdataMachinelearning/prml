\chapter{Sparse kernel machines}
\section{问题}
\begin{enumerate}
\item 为什么不能用解决线性回归的方式进行分类?
\item 当数据量很大的时候,计算所有核函数对很复杂,
如何只利用局部核解决问题? 即如何通过少量核求解问题?
\item 什么是无穷维?
\item 对偶表示的好处?
\item 为什么提出svm?
\item svm为什么需要正定核?
\item svm、rvm、type2极大似然估计、高斯过程的关系?
\item 对偶、非参方法、核?
\item 统计学习方法原始问题和对偶问题的解的大小?
\item 为什么最大间隔的分割面最好?
\item 为什么喜欢小的$w$,因为喜欢光滑的,即当x变化时,y的变化不大。
当需要光滑的曲线时,参数w应该是小的。
\end{enumerate}

\section{SVM核心思想}
引入核,通过升维的方式处理将线性不可分的问题变为
线性可分的问题。

\section{最大间隔分类器}
\subsection{损失函数}
\begin{equation}
L(w) = \sum_{n=1}^N E_\infty(y(x_n)t_n - 1) + \lambda||w||^2
\end{equation}
\section{重叠分类分布}

\subsection{损失函数}
\begin{equation}
L(w) = C\sum_{n=1}^N\xi_n + \frac{1}{2}||w||^2
\end{equation}

\section{龙星计划}
\subsection{损失函数}
\begin{equation} 
L(w) = L(w, b) + \lambda||w||^2
\end{equation}

\subsection{最大间隔与w的直观解释}
为什么最大间隔等价于最小化$||w||^2$。当w小的时候,
x变化引起y的变化较小,因此对于同样的y之间的间隔,
需要的x的间隔就会变大。
为什么要最大间隔呢:从几何上说,最大间隔可以减少
误分样本的个数,从而泛化能力好。
从光滑性的角度,间隔越大,函数越光滑,
我们需要小$||w||$,即我们希望x差别
小的时候,对应的y也相似。这一点可以从核的角度解释,
也可以和高斯过程建立联系。
稳定性,最大间隔比较稳定。
这里面包含着一个意思,最大间隔等价于最优光滑性,用数学表示
如下:
\begin{equation}
\underset{w}{\operatorname{arg max}}\frac{1}{w}
\equiv \underset{w}{\operatorname{arg min}}w^Tw
\end{equation}

\subsection{特征分解}
矩阵可以进行特征分解,
\begin{equation} 
\Sigma = \sum_{m=1}^M\lambda_mu_mu^T_m
\end{equation}

函数也可以进行特征分解,
\begin{equation} 
K(x, y) = \sum_{m=1}^\infty\gamma_m\phi_m(x)\phi_m(y)
\end{equation}
\begin{equation} 
f(x) = \sum_{m=1}^\infty\gamma_m\phi_m(x)
\end{equation}

\subsection{kernel函数}
用什么样的kernel函数,用什么样的特征,使用什么样的basic function,
使用什么样的正规则,加什么样的先验,实质都是一个事。
因此最大间隔就相当于建立了一个正则项。
对比岭回归中的正则项:
\begin{equation}
L(w) = \sum_{n=1}^N(t_n - w^Tx_n)^2 + \frac{1}{2}w^Tw
\end{equation}

\section{总结}
\begin{enumerate}
\item 泛化能力好,不容易过拟合
\cite{longxing2012machinelearning}
\end{enumerate}
\section{相关资料}
\begin{enumerate}
\item Andrew Ng 视频
\item 龙星计划
\item MLapp
\item 统计机器学习
\end{enumerate}
