\chapter{Linear models for regression}
\label{linear_models_regression}
\section{贝叶斯线性回归}
\subsection{参数分布}
公式符号与第\ref{probability_distribution}、
\ref{sparse_kernel_machines}章保持一致, 见Notation.

\begin{enumerate}
\item 参数后验分布一般形式
\begin{equation}
p(w|t) = N(w|\mu_N, \Sigma_N)
\end{equation}

\begin{equation}
\mu_N = \Sigma_N(\Sigma_0^{-1}m_0 + \beta \Phi^Tt)
\end{equation}

\begin{equation}
\Sigma_N^{-1} = \Sigma_0^{-1} + \beta\Phi^T\Phi
\end{equation}

\item 参数后验分布常用形式
\begin{equation}
p(w|t) = N(w|0, \alpha^{-1}I)
\end{equation}

\begin{equation}
\mu_N = \beta \Sigma_N\Phi^Tt 
\end{equation}

\begin{equation}
\Sigma_N^{-1} = \alpha I + \beta \Phi^T\Phi
\end{equation}

\end{enumerate}
参数分布验精度矩阵等于先验精度矩阵加数据精度矩阵

\section{经验贝叶斯}
同名经验贝叶斯,第二类极大似然估计、广义极大似然估计、模型近似

\begin{equation}
p(t|D) = \int\int\int p(t|w, \beta)p(w|\alpha, \beta)
p(\alpha,\beta|D)dwd\alpha d\beta
\end{equation}

近似预测

\begin{equation}
p(t|D) \simeq p(t|\hat{\alpha}, \hat{\beta}) = \int p(t|w, \hat \beta)
p(w|\hat \alpha, \hat \beta)dw
\end{equation}

\begin{equation}
\frac{dm_N}{d\alpha} = A^{-1}m_N
\end{equation}

\begin{equation}
\frac{dm_N^T}{d\alpha} = m_N^TA^{-1}
\end{equation}

\begin{equation}
\begin{aligned}
E(m_N) = \frac{\beta}{2}
(t^Tt - 2t^T\Phi m_N + m_N^T\Phi^T\Phi m_N) + \frac{\alpha}{2}m_N^Tm_N
= \frac{\beta}{2}t^Tt - \beta t^T\Phi m_N + \frac{1}{2}m_N^Tm_N \\
= \frac{\beta}{2}t^Tt - m_N^TAm_N + \frac{1}{2}m_N^TAm_N 
= \frac{\beta}{2}t^Tt - \frac{1}{2}m_N^TAm_N
\end{aligned}
\end{equation}

\begin{equation}
\begin{aligned}
dE(m_N) = -\frac{1}{2}dm_N^TAm_N = -\frac{1}{2}\beta^2dt^T\Phi A^{-1}AA^{-1}
\Phi^Tt = \frac{1}{2}\beta^2t\Phi A^{-1}A^{-1}\Phi^Tt
= \frac{1}{2}m_N^Tm_N
\end{aligned}
\end{equation}

\subsection{问题}
\begin{enumerate}
\item 极大似然使用条件
\end{enumerate}

\section{联系}
\begin{enumerate}
\item 经验贝叶斯、模型估计,贝叶斯线性回归,贝叶斯线性分类
、RVM回归,RVM分类,高斯过程
\end{enumerate}
