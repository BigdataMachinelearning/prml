\chapter{Question}
\section{问题}

\begin{enumerate}
\item PCA降维与隐变量降维之间的关系?\\
\item 矩阵稀疏什么时候好,什么时候坏?\\
一般稀疏的矩阵容易处理,网络社团算法一般都是在
稀疏矩阵中做的。对于推荐问题,稀疏以后就面临过拟合,推荐不准确的困难。
\item 似然函数的写法?\\
回归问题、产生式模型、判别式模型
\begin{equation}
p(D|w) = \prod^N_{n=1} p(y_n|w^Tx_n)
\end{equation}
分类问题
\begin{equation}
p(D|\theta) = \prod_{n=1}^N p(x_n, y_n|\theta)
= \prod_{n=1}^Np(x_n|\theta)p(y_n)
\end{equation}
\item 高斯过程图模型?\\
\item EM算法,广义EM算法,非参方法EM算法?\\
\end{enumerate}

\section{EM算法}
\begin{enumerate}
\item EM算法是否可以用于没有隐变量的情况?
\item 可不可以把隐变量当成参数使用梯度下降算法?
\item FA可以使用梯度下降算法吗?
\end{enumerate}
\section{Idea}
\begin{enumerate}
\item 熟悉的内容讲公式
\item 不熟悉的内容讲直观
\item 熟悉以后再讲公式
\item 组成一个加强组 
\end{enumerate}

\section{目标}
\begin{enumerate}
\item 发高质量论文
\item 至少精读一本书(PRML、MLAPP)
\item 集中阅读文献(略读)
\end{enumerate}

